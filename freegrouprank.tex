\documentclass{article}

% Package `amsthm` and `thmtools` must come before package `hyperref`.
\usepackage{amsthm}
\usepackage{thmtools}
% Package `hyperref` must come before package `complexity`.
\usepackage[pdftitle={The Folded Density Problem is P-complete}, pdfauthor={Jeffrey Finkelstein}]{hyperref}
\usepackage{complexity}
\usepackage{amsmath}
\usepackage{amssymb}

\declaretheorem[numberwithin=section]{theorem}
\declaretheorem[numberlike=theorem]{corollary}
\declaretheorem[numberlike=theorem]{lemma}
\declaretheorem[numberlike=theorem, style=definition]{definition}
\declaretheorem[numberlike=theorem, style=definition]{todo}

\newcommand{\FGR}{\textsc{Free Group Rank}}
\newcommand{\FD}{\textsc{Folded Density}}
\newcommand{\NDM}{\textsc{SOME RESTRICTED CLASS OF NFA-to-DFA Minimization}}
\newcommand{\PFolds}{\textsc{Parallel Folds}}
\newcommand{\gen}[1]{\langle #1 \rangle}
\newcommand{\Core}{\textnormal{Core}}
\newcommand{\pos}{\textnormal{pos}}
\newcommand{\nega}{\textnormal{neg}}
\newcommand{\Instance}{\textbf{Instance}: }
\newcommand{\Question}{\textbf{Question}: }

\author{Jef{}frey~Finkelstein}
\title{The Folded Density Problem is \P-complete}
\date{\today}

\begin{document}

\maketitle

\abstract{}

\section{Introduction}

The problem of computing the minimum rank of a finitely generated subgroup of a free group is \P-complete.
One way of computing the minimum rank of the subgroup is by computing the size of a Nielsen reduced set equivalent to the generating set of the subgroup.
The Nielsen reduction, however, is complicated and relatively inefficient.
It is more efficient to instead reduce the problem to a more easily solved graph problem.
A finitely generated subgroup of a free group has a natural representation as a directed, labeled graph.
Stallings' folding process \cite[Algorithm~5.4]{stallings83} for reducing a directed labeled graph is an analog of the Nielsen reduction for computing a minimal generating set for a free group.
In this work, we show that some computational problems concerning the execution of this process are \P-complete.

\section{Preliminaries}

If $S$ is an alphabet, we define the set of \emph{formal inverses} for $S$, denoted $S^{-1}$, by $\{ s^{-1} | s \in S \}$.
In this case, the symbol $s$ is the inverse of $s^{-1}$ and vice versa.
A \emph{word} over the alphabet $\Sigma$ is a finite sequence of symbols from $\Sigma$.
If $w$ is such a word, the \emph{length} of $w$, denoted $|w|$, is the length of its underlying sequence.
The word $w$ can be written as $s_1s_2\cdots s_n$ where each $s_i \in (S \cup S^{-1})$ where $|w| = n$ and $1 \leq i \leq n$.
If $u$ and $v$ are words over $S \cup S^{-1}$, where $u = u_1 u_2\cdots u_n$ and $v = v_1 v_2 \cdot v_m$ then the \emph{concatenation} of $u$ and $v$, denoted $uv$, is the word $u_1 u_2 \cdots u_n v_1 v_2 \cdots v_m$.
A word is \emph{freely reduced} if it contains no adjacent symbols which are inverses.
The \emph{free group} on $S$ is the set of freely reduced words on $S \cup S^{-1}$ under the operation which concatenates two words then freely reduces the result.
In this free group, a word $w$ where $w = w_1 w_2 \cdots w_{i - 1} s s^{-1} w_{i + 2} \cdots w_n$ (which is not freely reduced) is equivalent to the freely reduced word $w' = w_1 w_2 \cdots w_{i - 1} w_{i + 2} \cdots w_n$.
Each word over $S \cup S^{-1}$ has a unique, freely reduced representation. \cite{citationneeded}
We denote this freely reduced representative of the word $w$ by $\rho(w)$.
If $A$ is a set of words, we define $\rho(A)$ by $\rho(A) = \{ \rho(a) \, | \, a \in A\}$.
A \emph{finitely generated subgroup} of the free group on $S$ is the closure of a finite set of words over $S \cup S^{-1}$ under the free group operation.
If $U$ is a finite set of words, the subgroup generated by $U$ is denoted $\gen{U}$.
The \emph{rank} of a finitely generated subgroup is the cardinality of a generating set of minimum size.

In this work, we write ``directed graph'' instead of ``directed multigraph''; note that multiple, parallel edges are allowed.
If $G$ is a directed graph, we denote the vertex set of $G$ by $V(G)$ and the edge set of $G$ by $E(G)$.
If $e$ is an edge in a graph and $e = (u, v)$, where $u$ and $v$ are vertices, then the \emph{origin} of $e$, denoted $o(e)$, is defined by $o(e) = u$ and the \emph{terminus} of $e$, denoted $t(e)$, is defined by $t(e) = v$.
Two edges (in a multigraph) are \emph{parallel} if they have the same origin and terminus.
A \emph{path from $u$ to $v$} in $G$, or more simply, a \emph{path}, is a finite sequence of edges, $(e_1, e_2, \dotsc, e_n)$, such that $o(e_1) = u$, $t(e_n) = v$, and $t(e_i) = o(e_{i + 1})$ for all $i$ with $1 \leq i \leq n - 1$.
A \emph{rooted circle} is a path from ``root'' vertex $v \in V(G)$ to the same vertex, along which each intermediate vertex, except possibly $v$, is incident to exactly two edges (namely, the one preceding it in the path and the one following it in the path).
The directed graph $G$ is a \emph{flower graph} with respect to a vertex $v \in V(G)$ if it equals the union of circles rooted at $v$.
%% TODO better definitions of minimal spanning trees
If $G$ is a directed graph, a \emph{tree} $T$ with root $v$ is an acyclic subgraph of $G$ in which there is a path in $T$ from $v$ to every vertex in $V(T)$.
A tree $T$ with root $v$ is a \emph{spanning tree} for $G$ if there is a path in $T$ from $v$ to every vertex in $V(G)$.
A spanning tree is \emph{minimal} if no edges can be removed without violating the spanning property.
The \emph{additive density} of $G$, denoted $\delta(G)$, is defined by $\delta(G) = |E(G)| - |V(G)|$ for all graphs $G$.

We say $G$ is \emph{labeled} by $S$ if there exists a function $\mu \colon E(G) \to S$.
If $G$ is a labeled, directed graph, then its \emph{inverse graph}, denoted $\hat{G}$, has the same vertex set and edge set, but wherever there is an edge $(u, v) \in E(G)$ with label $s$ we add the inverse edge $(v, u)$ to $E(\hat{G})$ with label $s^{-1}$.
The \emph{positive} edges of $\hat{G}$ are the edges with labels in $S$ and the \emph{negative} edges are those with labels in $S^{-1}$.
If $E$ is a set of directed, labeled edges, we denote the positive edges in $E$ by $\pos(E)$ and the negative edges by $\nega(E)$.
The \emph{label} of a path $p$, denoted $\mu(p)$, is the sequence of labels of each of its constituent edges.
We sometimes abuse notation for the sake of convenience and discuss a path in $G$ when what we mean is a path in the subgraph induced by the positive edges of $\hat{G}$ in which edges $e$ with the inverse orientation are allowed, but their labels $\mu(e)$ are read as the corresponding inverse label $\mu(e)^{-1}$.
Analogous to freely reduced words in the free group on $S$, a \emph{reduced path} in $G$ is a path which does not contain any adjacent pairs of edges such that their labels are inverses.

If $G$ is a directed graph labeled by $S$ and $v \in V(G)$, then \emph{language of $G$ with respect to $v$}, denoted $L(G, v)$, is defined by
\begin{equation*}
  L(G, v) = \{ \mu(p) \, | \, p \text{ is a reduced path in } G \text{ from } v \text{ to } v\}.
\end{equation*}
The \emph{reduced language}, denoted $\rho(L(G, v))$, is defined by $\rho(L(G, v)) = \{ \rho(w) \, | \, w \in L(G, v) \}$.
The \emph{core} of a labeled, directed graph $G$ with respect to a vertex $v \in V(G)$, denoted $\Core(G, v)$, is the subgraph of $G$ induced by the set of all edges which are in some reduced path from $v$ to $v$.
A labeled, directed graph $G$ is a \emph{core graph} with respect to a vertex $v \in V(G)$ if $\Core(G, v) = G$.
A pair of edges is \emph{foldable} if they have the same label and they share an origin or a terminus.
A directed graph $G$ labeled by $S$ is \emph{folded} if there are no foldable edges.
In other words, for each vertex $v \in V(G)$ and each symbol $s \in S$, there is at most one edge in $E(G)$ with origin $v$ and label $s$ and there is at most one edge in $E(G)$ with terminus $v$ and label $s$.
Stated another way, $G$ is folded if there is no vertex $v \in V(G)$ such that $v$ has two (or more) in-edges with the same label or two (or more) out-edges with the same label.
Informally, a \emph{fold} is a function which identifies two foldable edges and their shared incident vertex.
%% TODO state that every graph has a folded representative, F(G) that produces
%% the same freely reduced language
\emph{Stallings' folding process} is a function, denoted $F$, which operates on labeled, directed graphs.
It proceeds by repeatedly choosing a foldable pair of edges and folding them until there are no more foldable edges.

We will utilize the following computational problems.
These should be thought of as optimization problems, but they are defined as decision problems in order to facilitate proofs of hardness.
\begin{definition}[\FGR]
  \setlength{\parindent}{0pt}
  \mbox{}

  \Instance finite alphabet $S$, finite set $U \subseteq (S \cup S^{-1})^*$, positive integer $k$.

  \Question Does $\gen{U}$ in the free group on $S$ have rank $k$ or less?
\end{definition}

\FGR{} is \P-complete \cite[Theorem~4.9]{am84} (see also \cite[Problem~A.8.11]{ghr95}).
We will use this fact to show the \P-completeness of some related problems.
In the following decision problems, we ask to compute some properties about the result of executing Stallings' folding process on a particular input.

\begin{definition}[\FD]
  \setlength{\parindent}{0pt}
  \mbox{}

  \Instance finite alphabet $S$, directed graph $G$, label function $\mu \colon E \to S$, integer $k$.
  %a vertex $v \in V(G)$ such that $G$ is a flower graph with respect to $v$ and a core graph with respect to $v$,

  \Question Does $F(G)$ have additive density at most $k$?
\end{definition}

\begin{definition}[\PFolds]
  \setlength{\parindent}{0pt}
  \mbox{}

  \Instance finite alphabet $S$, directed graph $G$, label function $\mu \colon E \to S$, integer $k$.
  %a vertex $v \in V(G)$ such that $G$ is a flower graph with respect to $v$ and a core graph with respect to $v$,

  \Question Is $N^\|_F(G)$ greater than or equal to $k$? Here, $N^\|_F(G)$ is the number of folds produced by running $F$ on input $G$ which are folds of a pair of parallel edges.
\end{definition}

\section{Results}

We first require a basic lemma comparing the number of vertices in a minimal spanning tree to the number of edges in the spanned graph.

\begin{lemma}\label{lem:tree}
  Let $G$ be a graph with vertex set $V$ and edge set $E$.
  If $T$ is a minimal spanning tree of $G$ then $|E(T)| = |V(G)| - 1$.
\end{lemma}
\begin{proof}
  This proof is by induction on the number of vertices in the graph.
  Let $T_n$ denote a minimal spanning tree on a graph $G_n$ with $n$ vertices.
  For the base case, consider the graph on one vertex with no edges.
  In this case, the minimal spanning tree $T_1$ is the empty tree, so
  \begin{equation*}
    0 = |E(T_1)| = |V(G)| - 1 = 1 - 1 = 0.
  \end{equation*}
  For the inductive step, suppose that for graphs with $i$ vertices, we have $|E(T_i)| = |V(G_i)| - 1$.
  Consider a minimal spanning tree $T_{i + 1}$ of a graph $G_{i + 1}$ on $i + 1$ vertices.
  Now the subgraph of $T_{i + 1}$ which is just $T_{i + 1}$ with a single leaf removed, call it $T_i$, is a minimal spanning tree of a subgraph of $G_{i + 1}$ which has $i$ vertices, call it $G_i$.
  By the inductive hypothesis, $|E(T_i)| = |V(G_i)| - 1$.
  This implies $|E(T_{i + 1})| = |E(T_i)| + 1 = |V(G_i)| = |V(G_{i + 1})| - 1$, which concludes the proof.
\end{proof}

We restate in the language of complexity theory a result implicit in \cite{km02}.
\begin{theorem}
  \FD{} is \P-complete, even when the graph is a both a flower graph and a core graph with respect to a distinguished vertex.
\end{theorem}
\begin{proof}
  Stallings' folding process \cite[Algorithm~5.4]{stallings83} is an algorithm which computes the function $F$; it can be implemented in $O(n \log^* n)$ time \cite{touikan06}.
  After computing $F(G)$, it is simple to compare the number of edges in the graph to the number of vertices plus $k$.
  Therefore, $\FD \in \P$.

  To show that \FD{} is \P-hard, we show a logarithmic space many-one reduction from the \P-complete decision problem \FGR{} to \FD.
  The reduction proceeds as follows on input $\langle S, U, k \rangle$ where $U$ is a finite set of words $\{u_1, u_2, \dotsc, u_m\}$ and for each $i$ with $1 \leq i \leq m$ we have $u_i = u_{i, 1}u_{i, 2}\dotsb u_{i, n}$ where $n$ is the length of $u_i$.
  Create and output a directed graph $G$ which is a flower graph with respect to a distinguished vertex $v_0$.
  There is one circle rooted at $v_0$ for each word $u$ in $U$, and each circle is divided into $|u|$ edges, one for each symbol of $u$.
  Wherever $u$ contains an inverse $s^-1$ of a symbol in $S$, that edge is reversed (that is, $(u, v)$ is replaced with $(v, u)$).
  Each edge of the circle is then labeled (by defining a labeling function $\mu$) with the corresponding symbol in $u$, or its inverse when the edge has been reversed.
  Output $\langle S, G, \mu, k - 1 \rangle$.
  Observe that the created graph is a flower graph with center $v_0$ and a core graph with respect to $v_0$.

  This function is computable in logarithmic space because a loop over the set $U$ requires $O(\log |U|)$ space, a loop over a word $u \in U$ requires $O(\log |u|)$ space, and writing the adjacency matrix of the graph and the corresponding labels can be done in constant space.
  It remains to show the correctness of the reduction.
  Our goal is to show that $\gen{U}$ has rank at most $k$ if and only if the graph $G$ produced by the reduction has the property that $\delta(F(G)) \leq k - 1$, or in other words, that $\delta(F(G)) + 1 \leq k$.

  First we examine the properties of the graph $G$ produced by the reduction.
  Let $G' = F(G)$ and $v'_0 = F_G(v)$, where $F_G(v)$ denotes the image of $v$ under the sequence of foldings produced by $F$.
  Now
  \begin{align*}
    \gen{U} & = \rho(L(G, v_0)) & \qquad \text{ (by \cite[Proposition~3.8]{km02})} \\
            & = \rho(L(G', v'_0)) & \qquad \text{ (by \cite[Lemma~3.4]{km02})} \\
            & = L(G', v'_0), & \qquad \text{ (by \cite[Lemma~2.9]{km02})}
  \end{align*}
  We know $G'$ has a spanning tree $T$ which is geodesic with respect to $v'_0$ by \cite[Lemma~6.6]{km02}.
  Let $[v'_0, z]_T$ denote the path from $v'_0$ to $z$ in $T$, for all vertices $z \in V(G')$.
  Let $[e]$ denote the path which results from the concatenation of paths $[v, o(e)]_T, e, [t(e), v]_T$ for all edges $e \in \pos(E(\hat{G'}) \setminus E(T))$ (in other words, $e$ is a positive edge in $\hat{G}$ not in the spanning tree of $T$).
  Let $Y_T = \{[e] \, | \, e \in \pos(E(\hat{G'}) \setminus E(T)) \}$.
  We know $L(G', v'_0) = \gen{Y_T}$ by \cite[Lemma~6.1]{km02}, so $\gen{U} = \gen{Y_T}$.
  Furthermore, $Y_T$ is Nielsen reduced by \cite[Proposition~6.7]{km02}, so the rank of $\gen{U}$ equals the rank of $\gen{Y_T}$, which equals $|Y_T|$.
  Observe now that
  \begin{align*}
    |Y_T| & = |\pos(E(\hat{G'}) \setminus E(T))| & \\
          & = |\pos(E(\hat{G'}))| - |\pos(E(T))| & \text{ (since } E(T) \subseteq E(\hat{G'}) \text{)} \\
          & = |E(G')| - |E(T)| & \text{ (by construction)} \\
          & = |E(G')| - (|V(G')| - 1) & \text{ (by \autoref{lem:tree})} \\
          & = |E(G')| - |V(G')| + 1 & \\
          & = \delta(G') + 1 & \text{ (by definition of } \delta \text{)}.
  \end{align*}
  Finally, we have that the rank of $\gen{U}$ equals $\delta(G') + 1$.
  With this equality, we conclude the proof by stating that $\gen{U}$ has rank at most $k$ if and only if $\delta(G') + 1 \leq k$, which is what we intended to show.
\end{proof}

Like several other \P-complete decision problems, \FD{} asks a question whose answer requires computing an inherently sequential function.
The inherently sequential part of this problem seems to be Stallings' folding process which consists of a sequence of steps, each of which depends on the result of the previous.

\begin{theorem}
  \PFolds{} is \P-complete, even when the graph is a both a flower graph and a core graph with respect to a distinguished vertex.
\end{theorem}
\begin{proof}
  Like for \FD, computing Stallings' folding process and counting the number of folds that occur on a pair of parallel edges places this problem in \P.
  We show a logarithmic space many-one reduction from \FD{} with graphs that are both flower graphs and core graphs with respect to a distinguished vertex to \PFolds{} with the same restriction to complete the proof.
  The reduction is the identity mapping on all components of the instance except for the integer $k$, which is mapped to $\delta(G) - k$.
  Counting the number of vertices and number of edges in $G$, then performing two subtractions can be performed in logarithmic space.
  It remains to show the correctness of the reduction.

  Each fold in Stallings' folding process decreases the number of edges in the final graph by one.
  Each fold decreases the number of vertices in the final graph by one, unless the fold operates on two parallel edges with the same label (that is, two edges with the same origin, terminus, and label), in which case, the number of vertices remains unchanged.
  (This includes folds of double self-loops.)
  Let $N_F(G)$ be the number of folds when Stallings' folding process is run on input $G$, and let $N^\|_F(G)$ be the number of those folds which are folds of a pair of a parallel edges.
  It follows that $|E(F(G))| = |E(G)| - N_F(G)$ and $|V(F(G))| = |V(G)| - (N_F(G) - N^\|_F(G))$.
  Hence
  \begin{align*}
    \delta(F(G)) & = |E(F(G))| - |V(F(G))| \\
                 & = (|E(G)| - N_F(G)) - (|V(G) - (N_F(G) - N^\|_F(G))) \\
                 & = (|E(G)| - |V(G)|) - N_F(G) + N_F(G) - N^\|_F(G) \\
                 & = \delta(G) - N^\|_F(G).
  \end{align*}
  Therefore, $\delta(F(G)) \leq k \iff \delta(G) - N^\|_F(G) \leq k \iff N^\|_F(G) \geq \delta(G) - k$.
\end{proof}

The labeled, directed graphs we are considering are very similar to nondeterministic finite automata (NFA).
In fact, a labeled, directed graph is folded if and only if the corresponding finite automaton is deterministic \cite[Remark~2.8]{km02}.
Intuitively, when the graph is interpreted as an NFA, the \FD{} problem asks if a minimal equivalent deterministic finite automaton (DFA) minimizes the difference between the number of transitions and the number of states.
This may be valuable if implementing a transition in some concrete implementation of a DFA is more expensive than implementing a state.
Similarly, the \PFolds{} problem asks if a minimal equivalent DFA has eliminated many redundant transitions.

Computing a minimal equivalent DFA from an arbitrary NFA is a problem of high computational complexity.
In general, computing a minimal DFA from an arbitrary NFA may take exponential time \cite{citationneeded}.
Indeed, even computing a minimal NFA from an arbitrary NFA that has a very limited amount of nondeterminism is \NP-hard \cite[Corollary~3]{bm08}.
Furthermore, no polynomial time $o(n)$-approximation algorithm exists for the size of a minimal equivalent NFA unless $\P = \PSPACE$ \cite[Theorem~4]{gs05}.

%% TODO Is this true? Check if this class of highly restricted NFAs is in the
%% class \deltaNFA defined in \cite{bm08}.
However, the good news is that Stallings' folding process essentially shows that computing a minimal NFA from a (FILL ME IN, SOME VERY RESTRICTED CLASS) NFA can be computed in polynomial time, and the minimal NFA is in fact deterministic!

\begin{definition}[\NDM]
  \setlength{\parindent}{0pt}
  \mbox{}

  \Instance nondeterministic finite automaton $M$, positive integer $k$.

  \Question Does there exist a deterministic finite automaton $M'$ with $k$ or fewer states such that $L(M) = L(M')$?
\end{definition}

\begin{theorem}
  $\NDM{} \in \P$.
\end{theorem}
%% TODO Use the techniques of recent works for analyzing NFAs with limited
%% nondeterminism, like \cite{bm08}, \cite{gs05}, and ``Measures of
%% Nondeterminism in Finite Automata'', to prove analagous results for free
%% group rank approximation for restricted types of subgroups.

\bibliographystyle{plain}
\bibliography{references}

\end{document}
