% Foreword

% context (focus on anyone) why now? - current situation, and why the need is so important
Allowing many processors to work in parallel offers no significant speedup when computing the rank of a finitely generated free group.
Stallings' folding process, an algorithm for simplifying a directed edge-labeled graph, when applied to those graphs that represent finitely generated free groups, provides a graph-theoretic method for computing the rank of that group.
% need (focus on readers) why you? - why this is relevant to the reader, and why something needed to be done
Does adding more processors to Stallings' folding process offer any significant speedup?
% task (focus on author) why me? - what was undertaken to address the need
We answer this question negatively, proving that Stallings' folding process itself is an inherently sequential algorithm.
% object (focus on document) why this document - what the document covers
This paper provides the proof along with the requisite mathematical background.

% Summary
%
% findings (focus on author) what? - what the work revealed when performing the task
We show how the problem of computing the rank of a finitely generated free group reduces to the problem of computing an arbitrary bit of the execution of Stallings' folding process, thereby showing that the process satisfies the definition of an ``inherently sequential algorithm''\kern-0.5em.
% conclusion (focus on readers) so what? - what the findings mean for the audience
Thus, although Stallings' folding process is computable in polynomial time, its asymptotic running time cannot be improved by adding more processors.
% perspective (focus on anyone) what now? - what should be done next
Since this process often appears in the theory of free groups, one may be able to use this fact to show that other problems and algorithms appearing in computational group theory are inherently sequential.
